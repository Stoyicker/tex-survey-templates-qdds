\chapter*{Horizontal response categories}
\question{{\bf Wie zufrieden sind Sie - alles in allem - mit der �ffentlichen Sicherheit und der Bek�mpfung der Kriminalit�t in Konstanz?}\\
\label{Horizontal response categories}
Bitte antworten Sie auf einer Skala von 0 bis 10 - wobei 0 bedeutet, dass Sie ganz und gar unzufrieden sind, und 10 bedeutet, dass Sie ganz und gar zufrieden sind. Mit Werten dazwischen k�nnen Sie Ihre Meinung abstufen.}

\hupAeleven{ganz und gar {\bf unzufrieden}}{ganz und gar {\bf zufrieden}}
\hupAten{ganz und gar {\bf unzufrieden}}{ganz und gar {\bf zufrieden}}
\hupAnine{ganz und gar {\bf unzufrieden}}{ganz und gar {\bf zufrieden}}
\hupAeight{ganz und gar {\bf unzufrieden}}{ganz und gar {\bf zufrieden}}
\hupAseven{ganz und gar {\bf unzufrieden}}{ganz und gar {\bf zufrieden}}
\hupAsix{ganz und gar {\bf unzufrieden}}{ganz und gar {\bf zufrieden}}
\hupAfive{ganz und gar {\bf unzufrieden}}{ganz und gar {\bf zufrieden}}
\newpage

\question{{\bf Wie zufrieden sind Sie - alles in allem - mit der �ffentlichen Sicherheit und der Bek�mpfung der Kriminalit�t in Konstanz?}\\
Bitte antworten Sie auf einer Skala von 0 bis 10 - wobei 0 bedeutet, dass Sie ganz und gar unzufrieden sind, und 10 bedeutet, dass Sie ganz und gar zufrieden sind. Mit Werten dazwischen k�nnen Sie Ihre Meinung abstufen.}

\hupBeleven{ganz und gar {\bf unzufrieden}}{ganz und gar {\bf zufrieden}}
\hupBten{ganz und gar {\bf unzufrieden}}{ganz und gar {\bf zufrieden}}
\hupBnine{ganz und gar {\bf unzufrieden}}{ganz und gar {\bf zufrieden}}
\hupBeight{ganz und gar {\bf unzufrieden}}{ganz und gar {\bf zufrieden}}
\hupBseven{ganz und gar {\bf unzufrieden}}{ganz und gar {\bf zufrieden}}
\hupBsix{ganz und gar {\bf unzufrieden}}{ganz und gar {\bf zufrieden}}
\hupBfive{ganz und gar {\bf unzufrieden}}{ganz und gar {\bf zufrieden}}
\newpage

\question{{\bf Wie zufrieden sind Sie - alles in allem - mit der �ffentlichen Sicherheit und der Bek�mpfung der Kriminalit�t in Konstanz?}\\
Bitte antworten Sie auf einer Skala von 0 bis 10 - wobei 0 bedeutet, dass Sie ganz und gar unzufrieden sind, und 10 bedeutet, dass Sie ganz und gar zufrieden sind. Mit Werten dazwischen k�nnen Sie Ihre Meinung abstufen.}

\hupCeleven{ganz und gar {\bf unzufrieden}}{ganz und gar {\bf zufrieden}}
\hupCten{ganz und gar {\bf unzufrieden}}{ganz und gar {\bf zufrieden}}
\hupCnine{ganz und gar {\bf unzufrieden}}{ganz und gar {\bf zufrieden}}
\hupCeight{ganz und gar {\bf unzufrieden}}{ganz und gar {\bf zufrieden}}
\hupCseven{ganz und gar {\bf unzufrieden}}{ganz und gar {\bf zufrieden}}
\hupCsix{ganz und gar {\bf unzufrieden}}{ganz und gar {\bf zufrieden}}
\hupCfive{ganz und gar {\bf unzufrieden}}{ganz und gar {\bf zufrieden}}
\newpage

\question{{\bf Wie zufrieden sind Sie - alles in allem - mit der �ffentlichen Sicherheit und der Bek�mpfung der Kriminalit�t in Konstanz?}\\
Bitte antworten Sie auf einer Skala von 0 bis 10 - wobei 0 bedeutet, dass Sie ganz und gar unzufrieden sind, und 10 bedeutet, dass Sie ganz und gar zufrieden sind. Mit Werten dazwischen k�nnen Sie Ihre Meinung abstufen.}

\hupDeleven{ganz und gar {\bf unzufrieden}}{ganz und gar {\bf zufrieden}}
\hupDten{ganz und gar {\bf unzufrieden}}{ganz und gar {\bf zufrieden}}
\hupDnine{ganz und gar {\bf unzufrieden}}{ganz und gar {\bf zufrieden}}
\hupDeight{ganz und gar {\bf unzufrieden}}{ganz und gar {\bf zufrieden}}
\hupDseven{ganz und gar {\bf unzufrieden}}{ganz und gar {\bf zufrieden}}
\hupDsix{ganz und gar {\bf unzufrieden}}{ganz und gar {\bf zufrieden}}
\hupDfive{ganz und gar {\bf unzufrieden}}{ganz und gar {\bf zufrieden}}
\newpage

\question{{\bf Wie zufrieden sind Sie - alles in allem - mit der �ffentlichen Sicherheit und der Bek�mpfung der Kriminalit�t in Konstanz?}\\
Bitte antworten Sie auf einer Skala von 0 bis 10 - wobei 0 bedeutet, dass Sie ganz und gar unzufrieden sind, und 10 bedeutet, dass Sie ganz und gar zufrieden sind. Mit Werten dazwischen k�nnen Sie Ihre Meinung abstufen.}

\hdownAeleven{ganz und gar {\bf zufrieden}}{ganz und gar {\bf unzufrieden}}
\hdownAten{ganz und gar {\bf zufrieden}}{ganz und gar {\bf unzufrieden}}
\hdownAnine{ganz und gar {\bf zufrieden}}{ganz und gar {\bf unzufrieden}}
\hdownAeight{ganz und gar {\bf zufrieden}}{ganz und gar {\bf unzufrieden}}
\hdownAseven{ganz und gar {\bf zufrieden}}{ganz und gar {\bf unzufrieden}}
\hdownAsix{ganz und gar {\bf zufrieden}}{ganz und gar {\bf unzufrieden}}
\hdownAfive{ganz und gar {\bf zufrieden}}{ganz und gar {\bf unzufrieden}}
\newpage

\question{{\bf Wie zufrieden sind Sie - alles in allem - mit der �ffentlichen Sicherheit und der Bek�mpfung der Kriminalit�t in Konstanz?}\\
Bitte antworten Sie auf einer Skala von 0 bis 10 - wobei 0 bedeutet, dass Sie ganz und gar unzufrieden sind, und 10 bedeutet, dass Sie ganz und gar zufrieden sind. Mit Werten dazwischen k�nnen Sie Ihre Meinung abstufen.}

\hdownBeleven{ganz und gar {\bf zufrieden}}{ganz und gar {\bf unzufrieden}}
\hdownBten{ganz und gar {\bf zufrieden}}{ganz und gar {\bf unzufrieden}}
\hdownBnine{ganz und gar {\bf zufrieden}}{ganz und gar {\bf unzufrieden}}
\hdownBeight{ganz und gar {\bf zufrieden}}{ganz und gar {\bf unzufrieden}}
\hdownBseven{ganz und gar {\bf zufrieden}}{ganz und gar {\bf unzufrieden}}
\hdownBsix{ganz und gar {\bf zufrieden}}{ganz und gar {\bf unzufrieden}}
\hdownBfive{ganz und gar {\bf zufrieden}}{ganz und gar {\bf unzufrieden}}
\newpage

\question{{\bf Wie zufrieden sind Sie - alles in allem - mit der �ffentlichen Sicherheit und der Bek�mpfung der Kriminalit�t in Konstanz?}\\
Bitte antworten Sie auf einer Skala von 0 bis 10 - wobei 0 bedeutet, dass Sie ganz und gar unzufrieden sind, und 10 bedeutet, dass Sie ganz und gar zufrieden sind. Mit Werten dazwischen k�nnen Sie Ihre Meinung abstufen.}

\hdownCeleven{ganz und gar {\bf zufrieden}}{ganz und gar {\bf unzufrieden}}
\hdownCten{ganz und gar {\bf zufrieden}}{ganz und gar {\bf unzufrieden}}
\hdownCnine{ganz und gar {\bf zufrieden}}{ganz und gar {\bf unzufrieden}}
\hdownCeight{ganz und gar {\bf zufrieden}}{ganz und gar {\bf unzufrieden}}
\hdownCseven{ganz und gar {\bf zufrieden}}{ganz und gar {\bf unzufrieden}}
\hdownCsix{ganz und gar {\bf zufrieden}}{ganz und gar {\bf unzufrieden}}
\hdownCfive{ganz und gar {\bf zufrieden}}{ganz und gar {\bf unzufrieden}}
\newpage

\question{{\bf Wie zufrieden sind Sie - alles in allem - mit der �ffentlichen Sicherheit und der Bek�mpfung der Kriminalit�t in Konstanz?}\\
Bitte antworten Sie auf einer Skala von 0 bis 10 - wobei 0 bedeutet, dass Sie ganz und gar unzufrieden sind, und 10 bedeutet, dass Sie ganz und gar zufrieden sind. Mit Werten dazwischen k�nnen Sie Ihre Meinung abstufen.}

\hdownDeleven{ganz und gar {\bf zufrieden}}{ganz und gar {\bf unzufrieden}}
\hdownDten{ganz und gar {\bf zufrieden}}{ganz und gar {\bf unzufrieden}}
\hdownDnine{ganz und gar {\bf zufrieden}}{ganz und gar {\bf unzufrieden}}
\hdownDeight{ganz und gar {\bf zufrieden}}{ganz und gar {\bf unzufrieden}}
\hdownDseven{ganz und gar {\bf zufrieden}}{ganz und gar {\bf unzufrieden}}
\hdownDsix{ganz und gar {\bf zufrieden}}{ganz und gar {\bf unzufrieden}}
\hdownDfive{ganz und gar {\bf zufrieden}}{ganz und gar {\bf unzufrieden}}

\newpage

\htextlinethree{stimme zu}{stimme nicht zu}{wei� nicht}

\htextlinefour{stimme zu}{teils teils}{stimme nicht zu}{wei� nicht}

\htextlinefive{stimme voll und ganz zu}{stimme eher zu}{stimme eher nicht zu}{stimme �berhaupt nicht zu}{wei� nicht}

\htextlinesix{stimme voll und ganz zu}{stimme eher zu}{teils teils}{stimme eher nicht zu}{stimme �berhaupt nicht zu}{wei� nicht}

\htextlineseven{stimme voll und ganz zu}{stimme eher zu}{teils teils}{teils teils}{stimme eher nicht zu}{stimme �berhaupt nicht zu}{wei� nicht}

\newpage

\htextthree{stimme zu}{stimme nicht zu}{wei� nicht}

\htextfour{stimme zu}{teils teils}{stimme nicht zu}{wei� nicht}

\htextfive{stimme voll und ganz zu}{stimme eher zu}{stimme eher nicht zu}{stimme �berhaupt nicht zu}{wei� nicht}

\htextsix{stimme voll und ganz zu}{stimme eher zu}{teils teils}{stimme eher nicht zu}{stimme �berhaupt nicht zu}{wei� nicht}

\htextseven{stimme voll und ganz zu}{stimme eher zu}{teils teils}{teils teils}{stimme eher nicht zu}{stimme �berhaupt nicht zu}{wei� nicht}

\newpage

\horizontalthree{\upthree{stimme zu}{}{stimme nicht zu}}{\downthree{1}{2}{3}}

\horizontalfour{\upfour{stimme zu}{}{}{stimme nicht zu}}{\downfour{1}{2}{3}{4}}

\horizontalfive{\upfive{stimme zu}{}{halb und halb}{}{stimme nicht zu}}{\downfive{1}{2}{3}{4}{5}}

\horizontalsix{\upsix{stimme zu}{}{}{}{}{stimme nicht zu}}{\downsix{1}{2}{3}{4}{5}{6}}

\horizontalseven{\upseven{stimme zu}{}{}{halb und halb}{}{}{stimme nicht zu}}{\downseven{1}{2}{3}{4}{5}{6}{7}}